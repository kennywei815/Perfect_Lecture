\documentclass{article}
\usepackage{fontspec}
\usepackage{xeCJK}        %%% Macros for Chinese/Japanese/Korean processing
\usepackage{amsmath}

\setCJKmainfont[AutoFakeBold=6,AutoFakeSlant=.4]{新細明體}
    %AutoFakeBold設定粗體字要多粗
    %AutoFakeSlant設定斜體字要多斜,範圍-0.999到0.999,負值為往左斜

    %以下四行非必要,但對於切換字型蠻好用的。
\defaultCJKfontfeatures{AutoFakeBold=6,AutoFakeSlant=.4} %以後不用再設定粗斜
\newCJKfontfamily\Kai{標楷體}        %定義指令\Kai則切換成標楷體
\newCJKfontfamily\Hei{微軟正黑體}    %定義指令\Hei則切換成正黑體
\newCJKfontfamily\NewMing{新細明體}  %定義指令\NewMing則切換成新細明體
    %註:若您的Windows有安裝別的字型,也可以自行設定。
\XeTeXlinebreaklocale "zh"
\XeTeXlinebreakskip = 0pt plus 1pt
\pagestyle{empty}

\begin{document}

\textbf{中文}

\textbf{粗體boldface 12345中文字{\Kai 楷粗}}正常Normal 12345中文字

\textit{斜體italic 12345中文字{\Hei 黑斜}}\:正常Normal 12345中文字

$\sqrt{2}_{\text{中文}}$  %%% With the "amsmath" package, now you can include CJK charaters in subscription and superscriptiion (in the "\text" tag)

\end{document}